\documentclass{article}

\usepackage[utf8x]{inputenc}
\usepackage[greek,english]{babel}
\usepackage{hyperref}
\newcommand{\en}{\selectlanguage{english}}
\newcommand{\gr}{\selectlanguage{greek}}



\title{Σύγκριση και ανάπτυξη ασαφών μέτρων ομοιότητας και συσχέτισης δεδομένων αισθητήρων σε εφαρμογές υγείας και περιβάλλοντα υποβοηθούμενης διαβίωσης.}

\author{Νίκος Τερλεμές}

\begin{document}
\gr
\maketitle

\section{Εισαγωγή}

Σκοπός αυτής της αναφοράς είναι να παρουσιάσει ένα συνοπτικό περίγραμμα της διπλωματικής και τα αποτελέσματα μέχρι τώρα.
Αυτή η διπλωματική στηρίζεται πάνω στο \en{Opportunity dataset}\gr, το οποίο περίεχει δεδομένα από ένα σύνολο αισθητήρων σε σενάρια αναγνώρισης ανθρώπινης δραστηριότητας.
Ο σκοπός της είναι να εξετάσει υπάρχοντα και να παράγει νέα μετρικά ομοιότητας μεταξύ σημάτων, τα οποία εντοπίζουν την κοινή δραστηριότητα ανεξάρτητα από το είδος του αισθητήρα και της δραστηριότητας.

\section{Μέθοδος}

Πέρα από τα γνωστά μετρικά ομοιότητας μεταξύ σημάτων \en(correlation, DTW,\gr κ.α), η διπλωματική ασχολείται κυρίως με μετρικά ομοιότητας μεταξύ ασαφών συνόλων και μεταξύ κατανομές πιθανοτήτων, τα οποία έχουν εξαχθεί από τα εν λόγω σήματα.

\subsection{Κατασκευή ασαφών συνόλων}
 
Η κατασκευή των ασαφών συνόλων των σημάτων βασίζεται στην κατανομή των τιμών τους, ανάλογα με την κατασκευή ενός ιστογράμματος ή μιας πιθανοτικής κατανομής. 
Υπάρχουν κάποιες παραμέτροι στο συγκεκριμένο βήμα, καθώς μπορούμε να κανονικοποιήσουμε το ασαφές σύνολο, να ορίσουμε το επίπεδο της λεπτομέρειας, να χρησιμοποιήσουμε κάποιον\en{kernel} \cite{WEBSITE:1} \gr κ.α.

\subsection{Ορισμός μετρικών ομοιότητας ασαφών συνόλων}

Ο σκοπός των μετρικών ομοιότητας είναι, δεδομένου ενός σήματος (και του αντίστοιχου ασαφούς συνόλου) που αντιστοιχεί σε μια δραστηριότητα, να υποδείξει ποια σήματα αντιστοιχούν στην ίδια δραστηριότητα.
Οπότε, ένα μετρικό ομοιότητας παίρνει σαν είσοδο 2 σήματα και παράγει 1 τιμή.
Προφανώς, οφείλει να ικανοποιεί ορισμένες προϋποθέσεις.\cite{basesimilarity} 
Μετά από βιβλογραφική έρευνα, αναπτύχθηκαν πάνω από 20 τέτοια μετρικά.

\begin{enumerate}
\item Κανονικοποιήμενο αθροισμά της τομής των 2 ασαφών συνόλων.\cite{truong1}
\item Κανονικοποιήμενο εσωτερικό γινόμενο της τομής των 2 ασαφών συνόλων και της τομής των 2 ασαφών συνόλων των παραγώγων των σημάτων.\cite{truong2}
\item Διάφορους συνδυασμούς πράξεων από την θεωρία συνόλων.\cite{similarityset}
\item Διάφορους συνδυασμούς πράξεων βασισμένων στους τελεστές συνεπαγώγης από την θεωρία ασαφών συνόλων.\cite{similarityset}
\end{enumerate}

\subsection{\en{Ground truth}}
\grΗ σύγκριση των διάφορων μετρικών, απαιτείται \en ground truth, \gr απέναντι στην οποία θα συκριθεί η κατάταξη των σημάτων που παράγει κάθε ένα.
Για αυτόν τον λόγο, ορίστηκαν πίνακες αποστάσεις μεταξύ των διάφορων είδων αισθητήρων, τοποθεσιών τους και δραστηριοτήτων.

\subsection{Σύγκριση κατατάξεων}

Η σύγκριση των 2 κατατάξεων (της ιδανικής και του μετρικού) γίνεται με βάση τo \en Degree of Ranking Accuracy (DOA)\cite{truong1}


\en
\bibliography{last_try} 
\bibliographystyle{ieeetr}
\end{document}

